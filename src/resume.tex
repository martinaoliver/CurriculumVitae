\documentclass{ExpressiveResume}
\usepackage{xcolor}
\usepackage{dashrule}
% ----- Resume -----
\begin{document}

% ----- Name + Contact Information -----
\resumeheader[
    firstname=Martina,
    lastname=Oliver Huidobro,
    email=martinaoliver1@gmail.com,
    phone=+447883738907,
    linkedin=martinaoliver,
    github=martinaoliver,
    city=London %TODO add nationality, settled status

]

\objective{Computational biologist with interest in mathematical modelling, AI and drug discovery.}

% ----- Education -----
\section{Education}

\experience{Mathematical Biology applied to Genetic Engineering}{PhD}{October 2020}{April 2024}{
    \noindent Imperial College London, Centre for Integrative Systems and Bioinformatics \hfill EPSRC Scholarship  \newline
    Focus on mathematical modelling and machine learning approaches to study gene expression patterns
}

\experience{Systems and Synthetic Biology}{MRes}{Oct 2019}{September 2020}{
    \noindent Imperial College London, Centre for Integrative Systems and Bioinformatics \hfill EPSRC Scholarship, Distinction 75\%  \newline
Focus on dynamical systems, programming and statistics in biology
}

\experience{Biochemistry}{BSc}{October 2016}{Oct 2019}{
    \noindent Imperial College London \hfill First Class Honours 71\%  \newline
    Focus in Bioinformatics, Systems Biology, Metabolic Network Engineering.
}

\experience{European Baccalaureate}{High School Diploma}{October 2016}{Oct 2019}{
    \noindent European School of Brussels I \hfill Highest degree of cohort, 92\% \newline
    Focus in Advanced Maths, Physics, Biology and Chemistry.
}


% ----- Work Experience -----
\section{Research Experience}

\experience{PhD thesis}{\href{https://github.com/martinaoliver/PhD_handover/blob/main/PhD_thesis.pdf}{Data-driven modelling of robust Turing patterns in synthetic biofilms}}{Jan 2019}{Sept 2024}{
    \achievement{
    Thesis title: "Data-driven modelling of robust Turing patterns in synthetic biofilms" \faLink"
    }
    \achievement{Developed and validated models of gene expression using mathematical tools such as PDE numerical solvers and stability analysis.}

\achievement{Used machine learning techniques: Bayesian inference and regression methods for parameter fitting; neural networks for image clustering; PCA and t-SNE for understanding the parameter space; and MCMC for system optimisation.}
\achievement{Generated and analysed large biological datasets using High Performance Computing clusters and multi-threading, and built SQL databases to store and query results.}

\achievement{ \href{https://papers.ssrn.com/sol3/papers.cfm?abstract_id=4733248}{Collaborated with experimental biologists to validate models and guide experimental design for the production of novel biomaterials.}}

\achievement{Speaker at \textbf{4 international conferences} in mathematical biology and received \textbf{2 best poster awards} (over 200 candidates).}

\achievement{\textbf{Published in 3 high-impact scientific journals}: \faLink \href{https://papers.ssrn.com/sol3/papers.cfm?abstract_id=4733248}{[1] Cell Systems}
            \href{https://doi.org/10.1111/1751-7915.13979}{[2] Journal of microbial biotech.} \href{https://www.biorxiv.org/content/10.1101/2024.09.09.611947v1}  {[3] PLOS ONE (in review)}
            \href{https://github.com/martinaoliver/PhD_handover/blob/main/PhD_thesis.pdf}{[4] Spiral PhD Thesis} and \href{https://www.sciencedirect.com/science/article/pii/S2405471222004367}{peer reviewed other publications [5].}


}


}

\experience{MRes Thesis}{Gene circuit optimization for tissue engineering}{Jan 2019}{Sept 2019}{
    \achievement{
        Thesis title: "Gene circuit optimization for tissue engineering "}
    \achievement{Parameter optimization}}


\experience{miCHIP}{miRNA detection method for pancreatic cancer diagnostics}{June 2018}{October 2018}{
    \achievement{Obtained proof of concept for an early-stage diagnostics method
    for Alzheimer’s, gaining experimental and computational experience with RNA synthesis and RNA structure prediction.}
    \achievement{Parameter optimization}}

\experience{BSc Thesis}{Molecular dynamics simulations of $\alpha$-synuclein in Parkinson's disease.}{Feb 2019}{July 2019}{
\achievement{
    Thesis title: "Molecular dynamics simulations of $\alpha$-synuclein in Parkinson's disease."}
\achievement{Predicted protein structure and dynamics using GROMACS software and pyMOL visualization.}}


\experience{Literature review}{Multiscale modelling of tumour-immune system interactions}{March 2018}{July 2018}{
    \achievement{
        Thesis title: ""}}



% ----- Technical Projects -----
\section{Professional Experience}

\experience{Nucleate}{Activator Lead}{March 2023}{Feb 2024}{
    \achievement{
        Organised events e.g. competition with UK biotech spinouts pitching for a £2M prize.}}

\experience{Imperial College London}{Teacher and Research Supervisor}{Oct 2020}{March 2024}{
    \achievement{Supervised 20 MSc and BSc tudents in mathematical and computational biology.}
    \achievement{Taught courses such as Programming for systems biology, Bioinformatics, Integrative systems biology, Maths.}}

\experience{miCHIP}{Co-founder}{Jan 2018}{--March 2019, Feb-April 2022}{
\achievement{, start-up for early-stage Alzheimer’s diagnostics}
    \achievement{As an undergrad, \textbf{secured £5k} funding and lab space to develop proof of concept.}

\achievement{Finalists in start-up competitions: FONS-MAD, WE Innovate, SynBioUK Catalyse. News highlights: \faLink \href{https://www.imperial.ac.uk/news/187629/four-student-ideas-that-could-change/}{[1]} \href{https://www.imperial.ac.uk/news/190372/five-women-led-startups-building-better-future/}{[2]}}}


\section{Relevant experience}

\section{Relevant courses}
\noindent\textbf{Genomic Data Science} John Hopkins University (current)\\
\textbf{Advanced Learning algorithms} Stanford University (current)\\
\textbf{Supervised machine learning} Stanford University - 2024

\section{Personal projects}
\experience{}{GenAI for efficient carbon-capturing enzymes}{September 2024}{ongoing}{
    \achievement{Fine-tuning LLM model to predict novel protein sequences with
    increased activity}
    \achievement{Fine-tuning HuggingFace model using modal cloud infrastructure.}}

\experience{2nd place at the EF Bio x AI Hackathon}{Biodiversity scoring}
 {November 2023}{}{
    \achievement{Combining diverse spatial datasets including endangered species,
        land type, genomic diversity to develop biodiversity metrics.}}

\experience{Convention on Biological Diversity}{Peer reviewer}{Checkdate}{}

\section{Hard Skills}
\noindent \textbf{Maths:} Differential equations, numerical methods, stability analysis, statistics\\
\textbf{Machine Learning:} Regression, Bayesian inference, PCA, t-SNE, MCMC optimization, LLMs, autoencoders\\
\textbf{Programming:} Python, R, MATLAB, Bash, High-Performance Computing, Cloud computing, SQL, Git, scikit-learn, TensorFlow, pyTorch, Biopython, GROMACS, pyMOL

\section{Soft Skills}
    \noindent Project Management, collaboration with experimentalists, team work, time management, self-teaching

\section{Languages}
\noindent Spanish (Fluent), English (Fluent), French (Fluent), Portuguese (Beginner)

\section{Publications \& Conferences}
\noindent \faFile*[regular] \textbf{Publications}
\vspace{0.2cm}
\begin{itemize}
    \small \item  \href{https://doi.org/10.1111/1751-7915.13979}{\textbf{Oliver Huidobro} et al. 2022. Synthetic spatial patterning in bacteria: advances based on novel diffusible signals. Microbial Biotechnology \faLink}
    \small \item \href{https://papers.ssrn.com/sol3/papers.cfm?abstract_id=4733248}{Tica, \textbf{Oliver Huidobro} et al. 2024. A Three-Node Turing Gene Circuit Forms Periodic Spatial Patterns in Bacteria. Cell Systems \faLink}
    \small \item  \href{https://doi.org/10.25560/111289}{\textbf{Oliver Huidobro} 2024. Data-driven modelling of robust Turing patterns in synthetic biofilms. PhD Thesis, Spiral \faLink}
    \small \item \textbf{Oliver Huidobro}, Endres 2024. Effects of multistability, absorbing boundaries and growth on
    Turing pattern formation. BioRxiv, in review at PLOS ONE
\end{itemize}


%\vspace{7pt}
\noindent\color{gray!50}\hdashrule[0.5ex]{\linewidth}{0.2pt}{1.5pt}
\vspace{0.2cm}

\noindent \faUsers \textbf{Conferences}

\noindent Speaker at:
\begin{itemize}
    \small \item Society for Math. Biology Conf. Seoul Univ. Korea. July 2024
    \small \item British Applied Maths Colloquium. Newcastle UK. April 2024
    \small \item London Mathematical Biology Conference. UCL, UK. Sept 2023
    \small \item CDT BioDesign Engineering Conference. ICL, UK. May 2023
\end{itemize}
\vspace{0.2cm}
Best Poster awards:
\begin{itemize}
    \small \item OKO International Symposium. Kyoto, Japan. August 2023
    \small \item Physics of Life Conference. Harrogate, UK. March 2023

\end{itemize}
\vspace{10pt}
\section{Personal Interests}
\noindent Passionate about surfing and sailing (Qualified RYA Coastal Skipper and French Sailing instructor).
\end{document}
